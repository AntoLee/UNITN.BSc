\documentclass[12pt]{article}

\usepackage{graphicx}
\usepackage{amssymb}
\usepackage{amsmath}
\usepackage[margin=1in]{geometry}
\usepackage{fancyhdr}
\usepackage{enumerate}
\usepackage[shortlabels]{enumitem}

\pagestyle{fancy}
\fancyhead[l]{Li Yifeng}
\fancyhead[c]{Homework \#4}
\fancyhead[r]{\today}
\fancyfoot[c]{\thepage}
\renewcommand{\headrulewidth}{0.2pt}
\setlength{\headheight}{15pt}

\newcommand{\bP}{\mathbb{P}}

\begin{document}
	
	\section*{Question 3}
	
	\noindent You have two unfair coins, one with the probability of heads equal to $p_1$ and the other with the probability of heads equal to $p_2$, where $p_2 \neq p_1$. In strategy A, you choose one coin at random and toss it twice. In strategy B, you toss both coins. What is the best strategy to maximize the probability of the event $E = \text{"the two tosses are both heads"}$?
	
	\begin{enumerate}[label={},leftmargin=0in]\item
		
		\subsection*{Solution}
			
			Let $\it{head}$ and $\it{tail}$ denote the results of a drop. Define the probability space as
		
			\[
			\begin{aligned}
				\Omega &= \{\mathrm{head},\,\mathrm{tail}\}\\
				\mathcal{F} &= \mathcal{P}(\Omega)\\
				\bP &:\enspace \text{the probability measure on $\mathcal{F}$}
			\end{aligned}
			\]
			
			Let $C_1$ and $C_2$ denote the events of selecting the first mentioned coin and the second mentioned coin, respectively, and let the first drop and the second drop be denoted by subscripts $1$ and $2$, respectively, then we have
			
			\[\bP_{C_{i_j}}(\{\mathrm{head}\}_k) = p_i,\quad i,j,k\in\{1,2\}\]
			
			\subsubsection*{Strategy A: choose one coin at random and toss it twice}
			
			By applying the principle of symmetry, we have
			
			\[\bP(C_{1_1}) = \bP(C_{2_1}) = \frac{1}{2}\]
			
			And from the context we know
			
			\[\bP_{C_{1_1}}(C_{1_2}) = \bP_{C_{2_1}}(C_{2_2}) = 1\]
			
			We know that $\{C_{1_1},C_{2_1}\}$ and $\{C_{1_2},C_{2_2}\}$ are two partitions of $\Omega$, then we have the probability of event $E$ in strategy A
			
			\[
			\begin{aligned}
				\bP(E_A) &= \bP(\{\mathrm{head}\}_1\cap \{\mathrm{head}\}_2)\\
				&= \sum_{j=1}^2\left(\bP(C_{j_1})\bP_{C_{j_1}}(\{\mathrm{head}\}_1)\bP_{C_{j_1}\cap \{\mathrm{head}\}_1}(\{\mathrm{head}\}_2)\right)\\
				&= \frac{{p_1}^2 + {p_2}^2}{2}
			\end{aligned}
			\]
			
			\subsubsection*{Strategy B: toss both coins}
			
			Because the order does not matter here, one can assume to drop the coin $C_1$ first with the coin $C_2$ second, that is
			
			\[\bP(C_{1_1}) = \bP(C_{2_2}) = 1\]
			
			Easy to see that $\{\mathrm{head}\}_1$ and $\{\mathrm{head}\}_2$ are stochastically independent, then we have the probability of event $E$ in strategy B
			
			\[
			\begin{aligned}
				\bP(E_B) &= \bP(\{\mathrm{head}\}_1\cap \{\mathrm{head}\}_2)\\
				&= \bP(\{\mathrm{head}\}_1)\bP(\{\mathrm{head}\}_2)\\
				&= \bP_{C_{1_1}}(\{\mathrm{head}\}_1)\bP_{C_{2_2}}(\{\mathrm{head}\}_2)\\
				&= p_1p_2
			\end{aligned}
			\]
			
			\subsubsection*{Conclusion}
			
			By applying AM–GM inequality, we have
			
			\[
			\begin{aligned}
				\frac{p_1 + p_2}{2}&\ge \sqrt{p_1p_2}\\
				&\Downarrow\\
				\frac{{p_1}^2 + {p_2}^2}{2}&\ge p_1p_2
			\end{aligned}
			\]
			
			Then we know
			
			\[\bP(E_A) \ge \bP(E_B)\]
			
		\subsection*{Answer}
		
			\[\boxed{\text{Strategy A is the best.}}\]
	
	\end{enumerate}
\end{document}

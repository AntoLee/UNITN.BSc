\documentclass[12pt]{article}

\usepackage{graphicx}
\usepackage{amssymb}
\usepackage{amsmath}
\usepackage[margin=1in]{geometry}
\usepackage{fancyhdr}
\usepackage{enumerate}
\usepackage[shortlabels]{enumitem}

\pagestyle{fancy}
\fancyhead[l]{Li Yifeng}
\fancyhead[c]{Homework \#5}
\fancyhead[r]{\today}
\fancyfoot[c]{\thepage}
\renewcommand{\headrulewidth}{0.2pt}
\setlength{\headheight}{15pt}

\newcommand{\bP}{\mathbb{P}}

\begin{document}
	
	\section*{Question 4}
	
	\noindent We are given a bowl that contains $6$ balls, numbered from $1$ to $6$. We extract two balls and denote $X$ and $Y$ the numbers on the ball obtained at the first (second) extraction, and $W = \max(X,Y)$ the maximum value obtained. In all scenarios, describe the probability distribution of $W$.
	
	\bigskip
	
	\begin{enumerate}[start=1,label={\bfseries Part \arabic*:},leftmargin=0in]
		\bigskip\item In the first scenario, assume that the extractions are made with replacement.
		
		\subsection*{Solution}
		
			By applying the principle of symmetry, easy to define the probability space as
			
			\[
			\begin{aligned}
				\Omega &= \{1,2,\dots,6\}\\
				\mathcal{F} &= \mathcal{P}(\Omega)\\
				\bP &:\enspace \bP(\{1\}) = \bP(\{2\}) = \dots = \bP(\{6\}) = \frac{1}{6}
			\end{aligned}
			\]
			
			Then as given, we have
			
			\[
			\begin{aligned}
				X,Y &:\enspace \Omega \rightarrow R\\
				W &= \max(X,Y)\\
				R &= \{1,2,\dots,6\}
			\end{aligned}
			\]
			
			Then we have the probability distribution of $W$ in scenario \#1 (denoted as $W_1$ and $p_1$)
			
			\[
			p_1(w_1) =
			\begin{cases}
				\begin{aligned}
					\bP(W_1 = 1) &= \bP(\{(1,1)\}) &= \frac{1}{6^2} = \frac{1}{36},&\quad w_1 = 1\\
					\bP(W_1 = 2) &= \bP(\{(1,2),(2,1),(2,2)\}) &= \frac{3}{36},&\quad w_1 = 2\\
					\bP(W_1 = 3) &= \bP(\{(1,3),(2,3),\dots,(3,3)\}) &= \frac{5}{36},&\quad w_1 = 3\\
					\bP(W_1 = 4) &= \bP(\{(1,4),(2,4),\dots,(4,4)\}) &= \frac{7}{36},&\quad w_1 = 4\\
					\bP(W_1 = 5) &= \bP(\{(1,5),(2,5),\dots,(5,5)\}) &= \frac{9}{36},&\quad w_1 = 5\\
					\bP(W_1 = 6) &= \bP(\{(1,6),(2,6),\dots,(6,6)\}) &= \frac{11}{36},&\quad w_1 = 6
				\end{aligned}
			\end{cases}
			\]
			
			Or a general formula without cases
			
			\[p_1(w_1) = \frac{2w_1-1}{6^2} = \frac{2w_1-1}{36},\quad w_1\in \{1,2,\dots,6\}\]
		
		\subsection*{Answer}
		
			\[\boxed{p_1(w_1) =
				\begin{cases}
					\begin{aligned}
						\frac{1}{36},&\quad w_1 = 1\\
						\frac{3}{36},&\quad w_1 = 2\\
						\frac{5}{36},&\quad w_1 = 3\\
						\frac{7}{36},&\quad w_1 = 4\\
						\frac{9}{36},&\quad w_1 = 5\\
						\frac{11}{36},&\quad w_1 = 6
					\end{aligned}
			\end{cases}}\]
		
		\bigskip\item In the second scenario, assume that the extractions are performed without replacement.
		
		\subsection*{Solution}
		
			Easy to see that
			
			\[
			p_2(w_2) =
			\begin{cases}
				\begin{aligned}
					\bP(W_2 = 2) &= \bP(\{(1,2),(2,1)\}) &= \frac{2}{6\times 5} = \frac{2}{30},&\quad w_2 = 2\\
					\bP(W_2 = 3) &= \bP(\{(1,3),(2,3),\dots,(3,2)\}) &= \frac{4}{30},&\quad w_2 = 3\\
					\bP(W_2 = 4) &= \bP(\{(1,4),(2,4),\dots,(4,3)\}) &= \frac{6}{30},&\quad w_2 = 4\\
					\bP(W_2 = 5) &= \bP(\{(1,5),(2,5),\dots,(5,4)\}) &= \frac{8}{30},&\quad w_2 = 5\\
					\bP(W_2 = 6) &= \bP(\{(1,6),(2,6),\dots,(6,5)\}) &= \frac{10}{30},&\quad w_2 = 6\\
				\end{aligned}
			\end{cases}
			\]
			
			Or a general formula without cases
			
			\[p_2(w_2) = \frac{2w_2-2}{6\times 5} = \frac{2w_2-2}{30},\quad w_2\in \{2,3,\dots,6\}\]
		
		\subsection*{Answer}
		
			\[\boxed{p_2(w_2) =
				\begin{cases}
					\begin{aligned}
						\frac{2}{30},&\quad w_2 = 2\\
						\frac{4}{30},&\quad w_2 = 3\\
						\frac{6}{30},&\quad w_2 = 4\\
						\frac{8}{30},&\quad w_2 = 5\\
						\frac{10}{30},&\quad w_2 = 6\\
					\end{aligned}
			\end{cases}}\]
			
		\bigskip\item In the third scenario, assume that after the first extraction, we replace the ball in the urn, together with another one with the same number.
		
		\subsection*{Solution}
		
			We know that
			
			\[p_{3_X}(x) = \bP(X = x) = \frac{1}{6},\quad x\in\{1,2,\dots,6\}\]
			
			Then we have
			
			\[
			\begin{aligned}
				p_{3_Y}(y) &= \bP(Y = y)\\
				&= \bP(X \ne y)\bP_{X \ne y}(Y = y) + \bP(X = y)\bP_{X = y}(Y = y)\\
				&= \frac{5}{6}\times \frac{1}{6-1+2} + \frac{1}{6}\times \frac{1+1}{6-1+2}\\
				&= \frac{1}{6}
			\end{aligned}
			\]
			
			\[y\in\{1,2,\dots,6\}\]
			
			With $p_{3_X} = p_{3_Y} = \frac{1}{6}$, easy to notice that scenario \#3 is identical as scenario \#1, which implies
			
			\[
			p_3(w_3) =
			\begin{cases}
				\begin{aligned}
					\bP(W_3 = 1) &= \bP(W_1 = 1) &= \frac{1}{36},&\quad w_3 = 1\\
					\bP(W_3 = 2) &= \bP(W_1 = 2) &= \frac{3}{36},&\quad w_3 = 2\\
					\bP(W_3 = 3) &= \bP(W_1 = 3) &= \frac{5}{36},&\quad w_3 = 3\\
					\bP(W_3 = 4) &= \bP(W_1 = 4) &= \frac{7}{36},&\quad w_3 = 4\\
					\bP(W_3 = 5) &= \bP(W_1 = 5) &= \frac{9}{36},&\quad w_3 = 5\\
					\bP(W_3 = 6) &= \bP(W_1 = 6) &= \frac{11}{36},&\quad w_3 = 6
				\end{aligned}
			\end{cases}
			\]
			
			Or a general formula without cases
			
			\[p_3(w_3) = \frac{2w_3-1}{36},\quad w_3\in \{1,2,\dots,6\}\]
		
		\subsection*{Answer}
		
			\[\boxed{p_3(w_3) =
				\begin{cases}
					\begin{aligned}
						\frac{1}{36},&\quad w_3 = 1\\
						\frac{3}{36},&\quad w_3 = 2\\
						\frac{5}{36},&\quad w_3 = 3\\
						\frac{7}{36},&\quad w_3 = 4\\
						\frac{9}{36},&\quad w_3 = 5\\
						\frac{11}{36},&\quad w_3 = 6
					\end{aligned}
			\end{cases}}\]
	\end{enumerate}
	
\end{document}
